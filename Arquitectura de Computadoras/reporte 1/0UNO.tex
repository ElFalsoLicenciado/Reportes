\documentclass{article}
\usepackage[export]{adjustbox}
\usepackage[spanish]{babel}
\usepackage[letterpaper,top=1.2cm,bottom=1.5cm,left=2cm,right=2cm,marginparwidth=1.5cm]{geometry}
\usepackage[strict]{changepage}
\usepackage[normalem]{ulem}
\usepackage[T1]{fontenc}
\usepackage{apacite,xurl,hyperref,pdfpages,graphicx,sectsty,ragged2e}
\usepackage[inkscapelatex=false]{svg}
\usepackage{float,array,listings,enumitem,subcaption,charter,amsfonts,amsthm,amsmath,mathtools}
\newtheorem{theorem}{Caso}
\sectionfont{\huge{}\underline\centering}
\subsectionfont{\LARGE{}\centering\textit}
\subsubsectionfont{\LARGE{}\centering\textbf}
\usepackage[affil-it]{authblk}
\renewcommand\Authfont{\fontsize{20}{25}\selectfont}
\usepackage[tablename=Tabla ]{caption}
\usepackage[figurename=Imagen ]{caption}
\newcommand{\R}{\mathbb{R}}
\newcommand{\tab}{$\quad$}

\begin{document}
% 
\includepdf{portada lab.pdf}
% 

\LARGE
\tableofcontents\newpage

\justifying

\Large

\section{Introducción}
\textbf{Circuito Lógico Combinacional (CLC):}

Definición:	circuito encargado de procesar (transformar) las señales binarias (información digital). \newline

Se puede representar como una «caja negra» (abstracción) con los siguientes componentes:

\begin{itemize}
    \item Una ó más entradas (señales binarias).
    \item Una ó	más salidas (señales binarias).
    \item Funcionalidad describiendo la relación entre entradas y salidas.
    \item \textit{Timing}: determina el retraso entre el cambio de una entrada y la respuesta de una salida.
\end{itemize}




\section{Desarrollo}

    \subsection{Parte 1 (\it{Sesión 12/09/2025})}

        \subsubsection{Sumador de 4 bits}


        \subsubsection{Restador de 2 bits}

    \subsection{Parte 2 (\it{Sesión 19/09/2025})}

        \subsubsection{Sumador y restador con complemento a 1 y 2}


\section{Conclusiones.}

\section{Bibliografía}
Abad, P., \& Torralbo, P. (n.d.). Tema 3. Circuitos Lógicos Combinacionales Sistemas Digitales \url{https://ocw.unican.es/pluginfile.php/2410/course/section/2423/tema_03.pdf}

‌
\end{document}